
\begingroup
\section{Afterword}

Hope you find \footahref{https://synrc.com/apps/n2o}{N2O},
\footahref{https://synrc.com/apps/kvs}{KVS}, and
\footahref{https://synrc.com/apps/mad}{MAD} stack small and concise.
Because it was a main goal during development.
We stay with minimal viable functionality criteria.

\paragraph{}
N2O is being free from unnecessary layers and code calls as much as possible.
At the same time it covers all you need to build
flexible web messaging relays using rich stack of protocols.

\paragraph{}
Minimalistic criteria allow you to see the system's
most general representation, which drives you to describe efficiently.
Focusing on core you could be more productive.
Erlang N2O and companion libraries altogether make
your life managing web applications easy without
efforts due to its naturally compact simple design and absence of code bloat.

\paragraph{}
You can see that perse\_transform is very usefull especially in javascript
protocol generation (SHEN), rest record-to-proplist generators (REST). So having
quote/unquote in language would be very useful. Fast and small
Erlang Lisp (LOL) is expecting compiler is this field as universal
Lisp-based macro system.

\paragraph{}
All apps in stack operate on its own DSL
records-based language: N2O --- \#action/\#element; KVS --- \#iterator/\#container.
This language is accessible directly from erlang languages: Joxa, Elixir, Erlang, Lol.

\paragraph{}
We hope that this book will guide you in the wild world of Erlang web development.
\endgroup
