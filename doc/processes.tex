\section{Erlang Processes}

\subsection{Reduced Latency}
The secret to reducing latency is simple. We try to deliver rendered HTML
as soon as possible and render JavaScript only when WebSocket initialization is complete.
It takes three steps and three Erlang processes to do that.

\includeimage{images/page-lifetime.png}{Page Lifetime}

\paragraph{}
Starting the page lifetime N2O uses HTTP process to serve the first HTML page.
After that it dies and spawns Transition process.
Then the browser initiates WebSocket connections to the similar URL endpoint.
N2O creates persistent WebSocket process and the Transition process dies.

\paragraph{}
Your page could also spawn processes with {\bf wf:async}.
These are persistent processes that act like regular Erlang processes.
This is a usual approach to organize non-blocking UI for file uploads and time other consuming operations.

\newpage
\subsection{Page Process}
The very first HTTP handler only renders HTML. All created
JavaScript actions are stored in the transition process.

\vspace{1\baselineskip}
\begin{lstlisting}
    transition(Actions) ->
        receive {'N2O',Pid} -> Pid ! Actions end.
\end{lstlisting}

\paragraph{}
HTTP handler dies immediately after returning HTML. Transition process
waits for a request from a WebSocket handler.

\subsection{Transition Process}
Right after receiving HTML the browser initiates WebSocket connection
thus starting WebSocket handler on the server. After responding with
JavaScript actions the Transition process dies and the only process left
running is WebSocket handler. At this point initialization phase is complete.

\subsection{Events Process}
After that all client/server communication is performed over
WebSocket channel. All events coming from the browser are
handled by N2O, which renders elements to HTML and actions to
JavaScript. Each user at any time has only one WebSocket process
per connection.

\subsection{Async Processes}
These are user processes created with {\bf wf:async} invocation.
This is a legacy name from the times when async technology
was called COMET for XHR channel. Async Processes are optional
and are only needed when you have a UI event taking too much
time to be processed, like gigabyte file uploads. You can create
many Async Processes per user.

\subsection{SPA Mode}
In SPA mode your N2O can serve no HTML at all. N2O elements are
bound during INIT handshake and thus can be used regularly as in DSL mode.

%\paragraph{}
%\includeimage{images/static-serving.png}{Static Serving}
\

