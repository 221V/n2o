\section{KVS: Abstract Erlang Database}

KVS is an Erlang abstraction over various native Erlang Key-Value databases,
like MBnesia. Its meta-schema includes only concept
of iterators (persisted linked lists) which locked or guarded by
containers (list head holders). All write operations to list are
passed through some Erlang process that guards the container. The application
which starts Erlang processes per container called \footahref{https://github.com/synrc/feeds}{feeds}.

\paragraph{}
The best use-case for KVS and Key-Value storages is to store operational data.
This data should back up to SQL warehouses for analysis. Operational data
should be scalable, protected, replicated, fail-tolerant, available. That is why
we store work-in-process data in Key-Value storages.

\paragraph{}
All write operations that are made to data with secondary indexes,
i.e. not like linked lists could be potentially handled without
feeds server. But some KV storages are not supporting secondary
indexes. Add those backends carefully. Currently kvs includes
following store backends: Mnesia, Riak and KAI.

\subsection{Polymorphic Records}

The data in KVS represented as plain Erlang records.
The first element of the tuple as usual indicates the name of bucket.
And the second element usually corresponds to the index key field.
Additional secondary indexes could be applied for stores that
supports 2i, e.g. kai, mnesia, riak.

\begin{lstlisting}
       Rec = {user,"maxim@synrc.com",[]}.

RecordName = element(1, Rec).
        Id = element(2, Rec).
\end{lstlisting}

\newpage
\subsection{Iterators}

All record could be chained into the double-linked lists in the database.
So you can inherit from the ITERATOR record just like that:

\begin{lstlisting}
-record(access, {?ITERATOR(acl),
    entry_id,
    acl_id,
    accessor,
    action}).
\end{lstlisting}

\begin{lstlisting}
    #iterator { record_name,
                id,
                version,
                container,
                feed_id,
                prev,
                next,
                feeds,
                guard }
\end{lstlisting}

This means your table will support add/remove linked list operations to lists.

\begin{lstlisting}
    1> kvs:add(#user{id="mes@ua.fm"}).
    2> kvs:add(#user{id="dox@ua.fm"}).
\end{lstlisting}

Read the chain (undefined means all)

\begin{lstlisting}
    3> kvs:entries(kvs:get(feed, user), user, undefined).
    [#user{id="mes@ua.fm"},#user{id="dox@ua.fm"}]
\end{lstlisting}

or just

\begin{lstlisting}
    4> kvs:entries(user).
    [#user{id="mes@ua.fm"},#user{id="dox@ua.fm"}]
\end{lstlisting}

Read flat values by all keys from table:

\begin{lstlisting}
    4> kvs:all(user).
    [#user{id="mes@ua.fm"},#user{id="dox@ua.fm"}]
\end{lstlisting}

\subsection{Containers}

If you are using iterators records this automatically
means you are using containers. Containers are just boxes
for storing top/heads of the linked lists. Here is layout
of containers:

\begin{lstlisting}
    #container { record_name,
                 id,
                 top,
                 entries_count }
\end{lstlisting}

\subsection{Extending Schema}

Usually you need only specify custom mnesia indexes and tables tuning.
Riak and KAI backends don't need it. Group you table into table packages
represented as modules with handle\_notice API.

\begin{lstlisting}
    -module(kvs_feed).
    -inclue_lib("kvs/include/kvs.hrl").

    metainfo() -> 
        #schema{name=kvs,tables=[
            #table{ name=feed,container=true,
                    fields=record_info(fields,feed)},
            #table{ name=entry,container=feed,
                    fields=record_info(fields,entry),
                    keys=[feed_id,entry_id,from]},
            #table{ name=comment,container=feed,
                    fields=record_info(fields,comment),
                    keys=[entry_id,author_id]} ]}.
\end{lstlisting}

And plug it into schema sys.config:

\begin{lstlisting}
    {kvs, {schema,[kvs_user,kvs_acl,kvs_feed,kvs_subscription]}},
\end{lstlisting}

And on database init

\begin{lstlisting}
    1> kvs:join().
\end{lstlisting}

It will create your custom schema.

\subsection{KVS API}

\subsection{Service}
System service functions of starting/stopping.

\vspace{1\baselineskip}
\begin{lstlisting}
-spec start() -> ok | {error,any()}.
-spec stop() -> stopped.
\end{lstlisting}
\vspace{1\baselineskip}

\subsection{Schema Change}
This API allows you to create, initialize and destroy the database schema.
Depending on database the format and/or featureset may differ.

\vspace{1\baselineskip}
\begin{lstlisting}
-spec destroy() -> ok.
-spec join() -> ok | {error,any()}.
-spec join(Node :: string()) -> [{atom(),any()}].
-spec init(Backend :: atom(), Module :: atom()) -> list(#table{}).
\end{lstlisting}
\vspace{1\baselineskip}

\subsection{Meta Info}
This API allows you to builds form from table metainfo.

\vspace{1\baselineskip}
\begin{lstlisting}
-spec modules() -> list(atom()).
-spec containers() -> list(tuple(atom(),list(atom()))).
-spec tables() -> list(#table{}).
-spec table(atom()) -> #table{}.
-spec version() -> {version,string()}.
\end{lstlisting}
\vspace{1\baselineskip}

\subsection{Chain Ops}
This API alows you to modify the data, chained lists.
You can create/1 the container, remove/add the the nodes from lists.

\vspace{1\baselineskip}
\begin{lstlisting}
-spec create(atom()) -> integer().
-spec remove(tuple()) -> ok | {error,any()}.
-spec remove(atom(), any()) -> ok | {error,any()}.
-spec add(tuple()) -> {ok,tuple()} | 
                                {error,exist} | 
                                {error,no_container}.
\end{lstlisting}
\vspace{1\baselineskip}


\subsection{Raw Ops}
This functions will patch the Erlang record inside database.

\vspace{1\baselineskip}
\begin{lstlisting}
-spec put(tuple()) -> ok | {error,any()}.
-spec delete(atom(), any()) -> ok | {error,any()}.
\end{lstlisting}
\vspace{1\baselineskip}

\subsection{Read Ops}
Allow you to read the Value by Key and list
recrods with given secondary indexes.

\vspace{1\baselineskip}
\begin{lstlisting}
-spec get(atom(), any()) -> {ok,any()} | 
                            {error,duplicated} |
                            {error,not_found}.
-spec get(atom(),any(), any()) -> {ok,any()}.
-spec index(atom(), any(), any()) -> list(tuple()).
\end{lstlisting}
\vspace{1\baselineskip}

\subsection{Import/Export}

\vspace{1\baselineskip}
\begin{lstlisting}
-spec load_db(string()) -> list(ok | {error,any()}).
-spec save_db(string()) -> ok | {error,any()}.
\end{lstlisting}
\vspace{1\baselineskip}
