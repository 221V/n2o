\section{API}

\subsection{Update DOM \bf{wf:update}}
Issues \#update{} action for an element.
It generates jQuery DOM update script and evaluates it.

\subsection{Wire JavaScript \bf{wf:wire}}
Issues \#wire{} action that sends JavaScript for evaluation on the client.

\subsection{Async Processes \bf{wf:comet}}
Creates Erlang processes that talk to primary page process by sending messages.
To redirect all updates and wire actions to page process {\bf wf:flush} should be called.
Usually you send messages to Async processes using N2O message bus.

\subsection{Message Bus {\bf wf:reg} and {\bf wf:send}}
N2O uses GProc process registry to manage pools of async processes.
It is used as PubSub message bus for N2O communications, later you can switch to robust RabbitMQ.
You can assign a process to the pool with {\bf wf:reg} and send message to a pool with {\bf wf:send}.

\subsection{Parse URL and Context parameters {\bf wf:q} and {\bf wf:qs}}
To extract url parameters or read from process context. {\bf wf:q} extracts variables
from context saved by controls postbacks and {\bf wf:qs} extracts variables from HTTP forms.

\subsection{Redirects {\bf wf:redirect}}
Redirects are not implemented with HTTP headers but using JavaScript action that changes {\bf window.location}.
It saves login context information and sends it in first packet after establishing WebSocket connection.

\subsection{Session Information {\bf wf:session}}
Stores any session information in ETS tables. Use {\bf wf:user}, {\bf wf:role} for
login and authorization purposes. See {\bf AVZ} library.

\subsection{Bridge information {\bf wf:header} and {\bf wf:cookie}}
You can read and issue cookie and headers information using internal Web-Server routines.
Also you can read peer IP with {\bf wf:peer}.

