\section{Elements}

\paragraph{}
With N2O you don't usually use HTML at all. Instead you define your page
in the form of Erlang records so that the page is type checked at compile time.
This is a classical CGI approach for compiled pages and it gives us all the benefits of
compile time error checking and provides DSL for client and server-side rendering.

\paragraph{}
By their nature Nitrogen elements are UI control primitives
that you use to construct Nitrogen pages with Erlang internal DSL.
They are rendered into HTML and JavaScript.
Behavior of all elements is controlled on server side and all the communication
between browser and server is performed over WebSocket channels.
Hence no POST requests and no HTML forms are used.

\subsection{Static Elements: HTML}
The core set of HTML elements includes br, headings, links, tables, lists and image.
Static elements will be transformed into HTML during rendering.

\paragraph{}
Static elements could also be used as placeholders for other HTML elements.
Usually ``static'' means elements that don't use postback parameter:

\vspace{1\baselineskip}
\begin{lstlisting}
    #textbox { id=userName, body= <<"Anonymous">> },
    #panel { id=chatHistory, class=chat_history }
\end{lstlisting}
\vspace{1\baselineskip}

This will produce the following HTML code:
\vspace{1\baselineskip}
\begin{lstlisting}
    <input value="Anonymous" id="userName" type="text"/>
    <div id="chatHistory" class="chat_history"></div>
\end{lstlisting}

\newpage
\subsection{Active Elements: HTML and JavaScript}
Also form elements exist that provide information for the server
and gather user input: button, radio and check buttons, text box area and password box.
Form elements usually allow assigning some Erlang postback handler to specify action behavior.
These elements are compiled into HTML and JavaScript. For example, during rendering some
Actions are converted to JavaScript and sent and executed in the browser.
Element definition specifies the list of {\bf source} elements providing data to event callback.

\vspace{1\baselineskip}
\begin{lstlisting}
    {ok,Pid} = wf:async(fun() -> chat_loop() end),
    #button { id=sendButton, body= <<"Send">>, 
              postback={chat,Pid}, source=[userName,message] }.
\end{lstlisting}
\vspace{1\baselineskip}

This will produce the following HTML:
\begin{lstlisting}
    <input value="Chat" id="sendButton" type="button"/>
\end{lstlisting}
and JavaScript code:
\begin{lstlisting}
    $('#sendButton').bind('click',function anonymous(event) { 
        ws.send(Bert.encodebuf({
            source: Bert.binary('sendButton'), 
            pickle: Bert.binary('g1AAAINQAAAAdX...'),
            linked: [
                Bert.tuple(Bert.atom('userName'),
                utf8.toByteArray($('#userName').val())),
                Bert.tuple(Bert.atom('message'),
                utf8.toByteArray($('#message').val()))] })); });
\end{lstlisting}
\vspace{1\baselineskip}

If postback action is specified the page module must include a callback to handle postback info:
\vspace{1\baselineskip}
\begin{lstlisting}
    event({chat,Pid}) ->
        wf:info("User ~p Msg ~p", [wf:q(userName),wf:q(message)]).
\end{lstlisting}
\vspace{1\baselineskip}

\newpage
\subsection{Base Element}
Each HTML element in N2O DSL have record compatibility with base element.

\vspace{1\baselineskip}
\begin{lstlisting}
    #element { ancestor=element,
               module,
               id,
               actions,
               class=[],
               style=[],
               source=[],
               data_fields=[],
               aria_states=[],
               body,
               role,
               tabindex,
               show_if=true,
               html_tag=Tag,
               title }.
\end{lstlisting}
\vspace{1\baselineskip}

Here {\bf module} is Erlang module which contains render function.
Data and Aria HTML custom fields are common attributes for all elements.
In case element name doesn't corresponds to HTML tag {\bf html\_tag} field provided.
As element contents {\bf body} field used for all elements.

\paragraph{}
Most HTML elements defined as base elements. You can choose even element
name that differs from HTML tag name:

\vspace{1\baselineskip}
\begin{lstlisting}
    -record(h6,    ?DEFAULT_BASE).
    -record(tbody, ?DEFAULT_BASE).
    -record(panel, ?DEFAULT_BASE_TAG(<<"div">>)).
    -record('div', ?DEFAULT_BASE_TAG(<<"div">>)).
\end{lstlisting}
\vspace{1\baselineskip}

\newpage
\subsection{DTL Template {\bf \#dtl}}
DTL stands for Django Template Language. DTL element lets you construct HTML
snippet from template with given placeholders for substitution.
Fields contains substitution bindings proplist, filename and templates folder.

\vspace{1\baselineskip}
\begin{lstlisting}
    -record(dtl, {?ELEMENT_BASE(element_dtl),
            file="index",
            bindings=[],
            app=web,
            folder="priv/templates",
            ext="html",
            bind_script=true }).
\end{lstlisting}
\vspace{1\baselineskip}

Consider we have prod.dtl file in priv/templates folder with two
placeholders \{\{title\}\}, \{\{body\}\} and default placeholder for JavaScript \{\{script\}\}.
All pleceholders except \{\{script\}\} should be specified in \#dtl element.
Here is an example how to use it:

\vspace{1\baselineskip}
\begin{lstlisting}
    body() -> "HTML Body".
    main() ->
      [ #dtl { file="prod", ext="dtl",
               bindings=[{title,<<"Title">>},{body,body()}]} ].
\end{lstlisting}
\vspace{1\baselineskip}

You can use templates not only for pages but also for controls. Let say we want
to use DTL iterators for constructing list elements:

\vspace{1\baselineskip}
\begin{lstlisting}[caption=table.html]
     <a href="{{i.url}}">{{i.name}}</a><br>
              <span>No items available :-(</span>
            
\end{lstlisting}
\vspace{1\baselineskip}

Here is example how to pass variables to this DTL template:

\vspace{1\baselineskip}
\begin{lstlisting}
    #dtl{file="table", bind_script=false, bindings=[{items,
      [ {[{name, "Apple"},     {url, "http://apple.com"}]},
        {[{name, "Google"},    {url, "http://google.com"}]},
        {[{name, "Microsoft"}, {url, "http://microsoft.com"}]} ]}]}.
\end{lstlisting}
\vspace{1\baselineskip}

For page templates bind\_script should be set for true, for control elements rendered from DTL,
bind\_script should be set to false.

\subsection{Button {\bf \#button}}

\paragraph{}
\begin{lstlisting}
    -record(button, {?ELEMENT_BASE(element_button),
            type= <<"button">>,
            name,
            value,
            postback,
            delegate,
            disabled}).
\end{lstlisting}

\paragraph{}
Sample:

\begin{lstlisting}
#button { id=sendButton, body= <<"Send">>,
          postback={chat,Pid}, source=[userName,message] }.
\end{lstlisting}

\subsection{Link {\bf \#dropdown}}

\begin{lstlisting}
    -record(dropdown, {?ELEMENT_BASE(element_dropdown),
            options,
            postback,
            delegate,
            value,
            multiple=false,
            disabled=false,
            name}).

    -record(option, {?ELEMENT_BASE(element_select),
            label,
            value,
            selected=false,
            disabled}).
\end{lstlisting}

\paragraph{}
Sample:

\begin{lstlisting}
    #dropdown { id=drop,
                value="2",
                postback=combo,
                source=[drop], options=[
        #option { label= <<"Microsoft">>, value= <<"Windows">> },
        #option { label= <<"Google">>,    value= <<"Android">> },
        #option { label= <<"Apple">>,     value= <<"Mac">> }
    ]},
\end{lstlisting}

\subsection{Link {\bf \#link}}

\paragraph{}
\begin{lstlisting}
    -record(link, {?ELEMENT_BASE(element_link),
            target,
            url="javascript:void(0);",
            postback,
            delegate,
            name}).
\end{lstlisting}
\vspace{1\baselineskip}

\subsection{Text Editor {\bf \#textarea}}

\paragraph{}
\vspace{1\baselineskip}
\begin{lstlisting}
    -record(textarea, {?ELEMENT_BASE(element_textarea),
            placeholder,
            name,
            cols,
            rows,
            value}).
\end{lstlisting}
\vspace{1\baselineskip}
