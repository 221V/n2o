\section{Elements}

\subsection{Page construction}
With N2O you don't usually use HTML at all. Instead you define your page
in the form of Erlang records so that the page is type checked at compile time.
This is a classical CGI approach for compiled pages and it gives us all the benefits of
compile time error checking and great server-side rendering.

\subsection{Control Elements}
By their nature Nitrogen elements are control primitives
that you use to construct Nitrogen pages with Erlang internal DSL.
They are rendered into HTML and JavaScript.
Behavior of all elements is controlled on server side and all the communication
between browser and server is performed over WebSocket channels.
Hence no POST requests and no HTML forms are used.

\subsection{Static Elements: HTML}
The core set of HTML elements includes br, headings, links, tables, lists and image.
Static elements will be transformed into HTML during rendering.

\paragraph{}
Static elements could also be used as placeholders for other HTML elements.
Usually ``static'' means elements that don't use postback parameter:

\vspace{1\baselineskip}
\begin{lstlisting}
    #textbox { id=userName, text= <<"Anonymous">> },
    #panel { id=chatHistory, class=chat_history }
\end{lstlisting}
\vspace{1\baselineskip}

This will produce the following HTML code:

\vspace{1\baselineskip}
\begin{lstlisting}
    <input value="Anonymous" id="userName" type="text"/>
    <div id="chatHistory" class="chat_history"></div>
\end{lstlisting}
\vspace{1\baselineskip}

\subsection{Active Elements: HTML and JavaScript}
Also form elements exist that provide information for the server
and gather user input: button, radio and check buttons, text box area and password box.
Form elements usually allow assigning some Erlang postback handler to specify action behavior.
These elements are compiled into HTML and JavaScript. For example, during rendering some
Actions are converted to JavaScript and sent and executed in the browser.
Element definition specifies the list of {\bf source} elements providing data to event callback.

\vspace{1\baselineskip}
\begin{lstlisting}
    {ok,Pid} = wf:comet(fun() -> chat_loop() end),
    #button { id=sendButton, text= <<"Send">>, 
              postback={chat,Pid}, source=[userName,message] },
\end{lstlisting}
\vspace{1\baselineskip}

This will produce the following HTML...

\vspace{1\baselineskip}
\begin{lstlisting}
    <input value="Chat" id="sendButton" type="button"/>
\end{lstlisting}
\vspace{1\baselineskip}

...and JavaScript code

\vspace{1\baselineskip}
\begin{lstlisting}
    $('#sendButton').bind('click',function anonymous(event) { 
        ws.send(Bert.encodebuf({
            source: Bert.binary('sendButton'), 
            pickle: Bert.binary('g1AAAINQAAAAdX...'),
            linked: [
                Bert.tuple(Bert.atom('userName'),
                utf8.toByteArray($('#userName').val())),
                Bert.tuple(Bert.atom('message'),
                utf8.toByteArray($('#message').val()))] })); });
\end{lstlisting}
\vspace{1\baselineskip}

If postback action is specified the page module must include a callback to handle postback info:

\vspace{1\baselineskip}
\begin{lstlisting}
    event({chat,Pid}) ->
        error_logger:info_msg("Username ~p Message ~p",
            [wf:q(userName),wf:q(message)]).
\end{lstlisting}
\vspace{1\baselineskip}

