\section{Erlang JavaScript/OTP Compiler}

\subsection{Compilation and Macros}
Erlang JavaScript/OTP Parse Transform has two modes driven
by {\bf \-jsmacro} and {\bf \-js} Erlang module attributes.
First mode is for precompiling Erlang module functions
to JavaScript strings, second is to export Erlang function
to a separate JavaScript file ready to run within Browser or node.js.

\paragraph{}
Sample usage of {\bf \-jsmacro} and {\bf \-js}:

\vspace{1\baselineskip}
\begin{lstlisting}
    -module(sample).
    -compile({parse_transform, shen}).
    -jsmacro([tabshow/0,doc_ready/1,event/3]).
    -js(doc_ready/1).
\end{lstlisting}

\subsection{Erlang Macro Functions}
Macro functions are useful when N2O is used as a server-side framework.
The function gets rewritten during Erlang compilation into a JavaScript format
string ready for embedding. Here is an example from N2O pages:

\begin{lstlisting}
    tabshow() ->
        X = jq("a[data-toggle=tab]"),
        X:on("show", 
            fun(E) -> T = jq(E:at("target")), tabshow(T:attr("href")) end).

    doc_ready(E) ->
        D = jq(document),
        D:ready(fun() -> T = jq("a[href=\"#" ++ E ++ "\"]"), T:tab("show") end).

    event(A,B,C) ->
        ws:send('Bert':encodebuf(
            [{source,'Bert':binary(A)}, {x,C},
             {pickle,'Bert':binary(B)}, {linked,C}])).

    main() ->
        Script1 = tabshow(),
        Script2 = event(1, 2, 3),
        Script3 = doc_ready(wf:js_list("tab")),
        io:format("tabshow/0:~n~s~nevent/3:~n~s~ndoc_ready/1:~n~s~n",
            [Script1,Script2,Script3]).
\end{lstlisting}
\vspace{1\baselineskip}

\newpage
Perform compilation and run tests:

\vspace{1\baselineskip}
\begin{lstlisting}
        $ erlc sample.erl
        $ erl
        > sample:main().
\end{lstlisting}
\vspace{1\baselineskip}

You will get following output:

\vspace{1\baselineskip}
\begin{lstlisting}
        tabshow/0:
            var x = $('a[data-toggle=tab]');
            x.on('show',function(e) {
                var t = $(e['target']);
                return tabshow(t.attr('href'));
            });

        event/3:
            ws.send(Bert.encodebuf({source:Bert.binary(1),
                                    x:3,
                                    pickle:Bert.binary(2),
                                    linked:3}));

        doc_ready/1:
        var d = $(document);
        d.ready(function() {
            var t = $('a[href="#' + 'tab' + '"]');
            return t.tab('show');
        });
\end{lstlisting}
\vspace{1\baselineskip}

You see, no source-map needed.

\subsection{JavaScript File Compilation}
Export Erlang function to JavaScript file with {\bf -js([sample/0,fun\_{args}/2])}.
You could include functions for both {\bf macro} and {\bf js} definitions.

\newpage
\subsection{Mapping Erlang/OTP to JavaScript/OTP}
Following libraries from OTP are partially supported for Erlang JavaScript Parse Transform: 
{\bf lists}, {\bf proplists}, {\bf queue}, {\bf string}.

\paragraph{\bf Example 1}\ 
\vspace{1\baselineskip}
\begin{lstlisting}
        S = lists:map(fun(X) -> X * X end,[1,2,3,4]),
\end{lstlisting}

Transforms to:

\begin{lstlisting}
        s = [1,2,3,4].map(function(x) {
            return x * x;
        });
\end{lstlisting}

\paragraph{\bf Example 2}\ 

\vspace{1\baselineskip}
\begin{lstlisting}
        M = lists:foldl(fun(X, Acc) -> Acc + X end,0,[1,2,3,4]),
\end{lstlisting}

Transforms to:

\begin{lstlisting}
        m = [1,2,3,4].reduce(function(x,acc) {
            return acc + x;
        },0);
\end{lstlisting}
