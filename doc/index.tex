\section{N2O: Web Framework for Erlang/OTP}

\paragraph{}
N2O was started as the first Erlang Web Framework
that uses WebSocket protocol only. We saved great compatibility with Nitrogen
and added many improvements, such as binary page construction,
binary data transfer, minimized process spawns, transmission of all events over the WebSocket
and work within Cowboy processes. N2O renders pages several times faster than Nitrogen.

\subsection{Wide Coverage}
N2O is unusual in that it solves problems in different web development domains
and stays small and concise at the same time. Started as a Nitrogen concept
of server-side framework it can also build client-side offline applications
using the same source code. This became possible with powerful Erlang JavaScript Parse
Transform which lets Erlang run on JavaScript platform and enables Erlang and JavaScript
interoperability. You can use Elixir, LFE and Joxa languages for backend development, as well.

\paragraph{}
N2O supports DSL and HTML templates. This lets you build JavaScript
control elements in Erlang and perform inline rendering with DSL using
the same code base for both client and server side.
How to use N2O is up to you. You can build mobile applications using server-side rendering
for both HTML and JavaScript thus reducing CPU cycles and saving the battery of a mobile device.
Or you can create rich offline desktop applications using Erlang JavaScript compiler.

\newpage
\subsection*{Why Erlang in Web?}
We have benchmarked all the existing modern web frameworks built using functional
languages and Cowboy was still the winner. The chart below shows raw HTTP
performance of functional and C-based languages with concurrent
primitives (Go, D and Rust) on a VAIO Z notebook with i7640M processor.

\includeimage{n2o_benchmark.png}{Web-Servers raw performance grand congregation}

\paragraph{}
Erlang was built for low latency streaming of binary data in telecom systems.
It's fundamental design goals included high manageability, scalability
and extreme concurrency. Thinking of WebSocket channels as binary
telecom streams and web pages as user binary sessions
helps to get an understanding of the choice
of Erlang over other alternatives for web development.

\paragraph{}
Using Erlang for web allows you to unleash the full power of telecom systems for
building web-scale, event-driven, message-passing, NoSQL, asynchronous, non-blocking,
reliable, highly-available, high-performance, secure, real-time, distributed applications.
See Erlang: The Movie II.

\paragraph{}
N2O outperforms full Nitrogen stack with only 2X raw HTTP Cowboy
performance downgrade thus upgrading rendering performance several
times compared to any other functional web framework. And
sure it's faster than raw HTTP performance of Node.js.

\subsection{Rich and Lightweight Applications}
There exists two approaches for designing client/server communication.
The first one is called 'data-on-wire'. With this approach only JSON, XML or binary
data are transferred over RPC and REST channels. All HTML rendering
is performed on the client-side. This is the most suitable approach for building desktop
applications. Examples include Chaplin/CoffeeScript, Meteor
and ClojureScript. This approach can also be used for building mobile clients.

\paragraph{}
Another approach is sending pre-rendered parts of pages and JavaScript
and then replacing HTML and executing JavaScript on the client-side. This approach
is better suited for mobile web development since the
client doesn't have much resources.

\paragraph{}
With N2O you can create both types of applications. You can use N2O REST framework
for desktop applications based on Cowboy REST API along with DTL
templates for initial HTML renderings for mobile applications.
You can also use Nitrogen DSL-based approach for modeling parts of pages
as widgets and control elements, thanks to rich
Nitrogen elements collections provided by Nitrogen community.

\paragraph{}
In cases where your system is built around Erlang infrastructure, N2O
is the best choice for fast web prototyping, simplicity
of use and clean codebase. Despite HTML transfers over the wire,
you will still have access to all your Erlang services directly.

\paragraph{}
You can also create offline applications using Erlang JavaScript compiler
just the way you would use ClojureScript, JScala, Elm, WebSharper
or any other similar tool. N2O includes: REST micro-frameworks,
server-side and client-side rendering engines,
WebSocket events streaming, JavaScript generation
and JavaScript macro system along with {\bf AVZ} authorization
library (Facebook, Google, Twitter, Github, Microsoft), key-value storages
access library {\bf KVS} and {\bf MQS} Message Bus client library (GProc, RabbitMQ).

\subsection{JSON and BERT}
N2O uses JSON and BERT. All messages passed over
WebSockets are encoded in native Erlang External Term Format.
It is easy to parse it in JavaScript with {\bf dec(msg)}
and it helps to avoid complexity on the server side. Please refer
to \footahref{http://bert-rpc.org}{http://bert-rpc.org} for detailed information.

\subsection{DSL and Templates}
We liked Nitrogen for the simple and elegant way it constructs typed
HTML with internal DSL. This is analogous to Scala Lift,
OCaml Ocsigen and Haskell Blaze approach. It lets you develop reusable control
elements and components in the host language.

\paragraph{}
Template-based approach (Yesod, ASP, PHP, JSP, Rails, Yaws and ChicagoBoss)
requires programmers to deal with raw HTML. It allows
defining pages in terms of top-level controls, placeholders
and panels. Thus N2O combines both approaches.

\paragraph{}
The main N2O advantage is combining both fast prototyping and suitability for large
scale projects. Below is an example of complete Web Chat implementation using
WebSockets. It demonstrates Templates, DSL and asynchronous
inter-process communication.

\newpage
\begin{lstlisting}[caption=chat.erl]
    -module(chat).
    -compile(export_all).
    -include_lib("n2o/include/wf.hrl").

    main() -> 
      [ #dtl { file="index", 
               bindings=[{title,"Chat"},{body,body()}] } ].

    body() ->
        {ok,Pid} = wf:async(fun() -> loop() end),

      [ #span    { id=title,       body="Your chatroom name: " }, 
        #textbox { id=userName,    body="Anonymous" },
        #panel   { id=chatHistory, class=chat_history },
        #textbox { id=message },
        #button  { id=sendButton,  source=[userName,message],
                                   body="Send", 
                                   postback={chat,Pid} } ].

    event({chat,Pid}) ->
        Username = wf:q(userName),
        Message = wf:q(message),
        wf:reg(room),
        Pid ! {message, Username, Message}.

    loop() ->
        receive { message, Username, Message} ->
                  Terms = [ #span { body=Username }, ": ",
                            #span { body=Message }, #br{} ],
                  wf:insert_bottom(chatHistory, Terms),
                  wf:flush(room) end, loop().
\end{lstlisting}

You should try to build that with your favorite language/framework!

\subsection*{Changes from Nitrogen}
We feel free to break some compatibility with the original
Nitrogen framework, mostly because we want to have a clean codebase
and achieve the best performance. However, it's still possible to easily port
Nitrogen web sites to N2O. E.g., N2O returns id and class semantics
of HTML and not {\bf html\_id}.

\paragraph{}
We simplified HTML rendering by not using
{\bf html\_encode} which should be handled by application layer.

\paragraph{}
We removed Nitrogen.js originally created by Rusty Klophaus
because of the pure WebSocket nature of N2O which doesn't
require XHR helper methods on the client-side. We added XHR fallback
handling by using 'Bullet' library written by Loïc Hoguin.

\paragraph{}
We also removed {\bf simple\_bridge} and optimized N2O on each level to
achieve maximum performance and simplicity. We hope you will enjoy
using N2O. We are fully convinced it is the most efficient way to
build Web applications in Erlang.

\paragraph{}
Original Nitrogen was tested in production under high load and we
decided to remove {\bf nprocreg} process registry along 
with {\bf{action\_comet}} heavy process creation. N2O creates a single
process for async WebSocket handler, all async operations
are handled within Cowboy processes.

\paragraph{}
Also, we introduced new levels of abstraction. You can extend
protocols (Nitrogen, Heartbeat, Binary), change protocol formatters (BERT,
JSON, MessagePack, BED), inject your code on almost any level. The code structure
is clean and Nitrogen compatibility layer is detachable from N2O.
