\section{Universal Web Framework for Erlang/OTP}

\subsection*{Overview}
Nitrogen 2 Optimized, N2O was started as the first Erlang Web Framework
that relies completely on WebSocket transport. Great compatibility with Nitrogen
was retained and many improvements were made, such as binary page construction,
binary data transfer, all events transmitted over WebSocket channel, minumum process spawns,
work within Cowboy processes. N2O page renderer is several times faster
than the original one.

\subsection*{Wide Coverage}
N2O is an unusual framework aiming to solve problems at different web development domains
while staying small and concise at the same time. Started off as a Nitrogen concept
of server-side framework it can also build client-side offline applications
using same site sources. This becomes possible with the powerful Erlang JavaScript Parse
Transform which extends Erlang language to JavaScript platform with two-side
interoperability. You can as well use Elixir, LFE and Joxa languages to extend
server parts of your infrastructure.

\paragraph{}
It also combines DSL and HTML templates approaches allowing to build JavaScript
control elements using Erlang languge and use inline rendering with DSL using
the same code for both client and server sides!
You decide how to use N2O, whether for mobile applications development using server-side rendering
for HTML and JavaScript saving CPU power on mobile devices or for creating rich desktop
offline applications with Erlang JavaScript compiler.

\subsection*{Why Erlang in Web?}
We've measured all the existing modern web frameworks built with functional
languages and Cowboy is still the king. On the chart below you can see raw HTTP
performance of functional and C-like languages with concurrent
primitives (Go, D and Rust) on VAIO Z notebook with i7640M processor.

\includeimage{connections.png}{Web-Servers raw performance grand congregation}

\newpage
\paragraph{}
Erlang was built for streaming binary data for telecom systems with
low latency. It is created with extreme manageability, scalability
and concurrent processing in mind. Assuming WebSockets channels are binary
telecom streams and web pages are user binary sessions
helps for better undestanding the choice of Erlang for Web.

\paragraph{}
Using Erlang for Web you unlock the full power of telecom systems for
web-scale, asynchronuous, non-blocking, sharded, event-driven,
message-passing, noSQL, reliable, highly-available, high-performance,
secure, qualative, real-time, clusterable applications. See Erlang: The Movie II.

\paragraph{}
N2O outperforms full Nitrogen stack with only 2X downgrade of raw
HTTP Cowboy performance thus rising rendering performance several
times in comparison to any other functional web framework and for
sure it is faster than raw HTTP node.js performance.

\subsection*{Rich and Lightweight Applications}
There are two approaches to design communications between client/server.
The first one is called 'data-on-wire', where only JSON, XML or Binary
data are transfered on the channel through RPC, REST. All HTML rendering
is done on the client side. This is mostly suitable for desktops
applications. The examples are Chaplin/CoffeeScript, Meteor/JavaScript
and ClojureScript.

\paragraph{}
The other approach is to send pre-rendered parts of pages and javascript,
and only replace HTML parts and execute JavaScript on the client side. This approach
is better suited for mobile applications, where client doesn't have much resources.

\paragraph{}
With N2O you can create both types of applications: using N2O REST framework
for first type of application based on Cowboy REST API along with DTL
templates for initial HTML renderings. And also Nitrogen DSL-based approach
to model parts of the pages as widgets and control elements thanks to rich
Nitrogen elements collections provided by Nitrogen community. 

\paragraph{}
In the cases where your system is built around Erlang infrastructure, N2O
is the best choice that you can make for fast web prototyping, simpicity
of use and codebase clearity. Despite HTML tranfer over the wire,
you will still have access to all your erlang services directly.

\paragraph{}
Also you can create off-line clients using Erlang JavaScript compiler
like you do with ClojureScript, JScala, Elm, WebSharper or similar
technologies. N2O covers: REST micro-frameworks, server-side rendering,
client-side rendering, WebSocket event streaming, JavaScript generation
and JavaScript macro system along with {\bf AVZ} authorization
library (Facebook, Google, Twitter, Github, Microsoft), key-value storages
access library {\bf KVS} and {\bf MQS} Message Bus client library (GProc, RabbitMQ).

\subsection*{DSL and Templates}
We liked Nitrogen for simple and elegant way of typed HTML page
construction with DSL, based on the host language as it is done in Scala Lift,
OCaml Ocsigen and Haskell Blaze. It helps to develop re-usable control
elements and components using the host language.

\paragraph{}
Template-based approach pushes programmers to deal with raw HTML,
like Yesod, ASP, PHP, JSP, Rails, Yaws, ChicagoBoss. It helps to
define the page in terms of top-level controls, playholders
and panels. So N2O combines both approaches.

\paragraph{}
Main N2O attraction is the fast prototyping. We also use it in large
scale projects. Here is the complete Web Chat example working with
WebSockets that demonstrates the use of Templates, DSL and async
interprocesses communications:

\subsection*{JSON and BERT}
N2O is able to use JSON and BERT. All messages that passed over
WebSockets are encoded with native Erlang External Term Format
which is easily parsed in JavaScript with {\bf Bert.decode(msg)}
and helps to avoid complexity on server-side. Please refer
to \footahref{http://bert-rpc.org}{http://bert-rpc.org} for more information.

\newpage
\begin{lstlisting}[caption=chat.erl]
-module(chat).
-compile(export_all).
-include_lib("n2o/include/wf.hrl").

main() -> 
    [ #dtl{file = "index", bindings=[{title,"Chat"},{body,body()}]} ].

body() -> %% area of http handler
    {ok,Pid} = wf:comet(fun() -> chat_loop() end),
  [ #span { text= <<"Your chatroom name: ">> }, 
    #textbox { id=userName, text= <<"Anonymous">> },
    #panel { id=chatHistory, class=chat_history },
    #textbox { id=message },
    #button { id=sendButton, text= <<"Send">>, 
              postback={chat,Pid}, source=[userName,message] },
    #panel { id=status } ].

event({chat,Pid}) -> %% area of websocket handler
    Username = wf:q(userName),
    Message = wf:q(message),
    Terms = [ #span { text="Message sent" }, #br{} ],
    wf:insert_bottom(chatHistory, Terms),
    wf:reg(room),
    Pid ! {message, Username, Message};

event(Event) -> error_logger:info_msg("Event: ~p", [Event]).

chat_loop() -> %% background worker ala comet
    receive 
        {message, Username, Message} ->
            Terms = [ #span { text=Username }, ": ",
                      #span { text=Message }, #br{} ],
            wf:insert_bottom(chatHistory, Terms),
            wf:flush(room);
        Unknown -> 
            error_logger:info_msg(
                "Unknown Looper Message ~p",[Unknown])
    end,
    chat_loop().
\end{lstlisting}

And try to build the similar functionality with your favourite language/framework.

\subsection*{Changes from Nitrogen}
We feel free to brake some of the compatability with the original
Nitrogen framework, mostly because we want to have a clean codebase
and fastest speed. However, it is still possible to easily port
Nitrogen web sites to N2O. E.g. N2O returns id and class semantics
of HTML and not {\bf html\_id}.

\paragraph{}
We simplified HTML rendering by not using
{\bf html\_encode} which should be handled by the application layer.

\paragraph{}
Nitrogen.js that was originally created by Rusty Klophaus,
was removed because of pure WebSocket nature of N2O which does not
require XHR helper methods on the client. We added XHR fallback
handling by using 'Bullet' library written by Loïc Hoguin.

\paragraph{}
We also dropped {\bf simple\_bridge} and optimized N2O at every level to
archive the maximum performance and simplicity. We hope you will enjoy
using N2O and we are confident that it is the most productive way to
build Web applications with Erlang.

\paragraph{}
Original Nitrogen was tested in production under high-load and we
decided to drop out {\bf nprocreg} process registry along 
with {\bf{action\_comet}} heavy process creation. N2O now creates
only one process for async websocket handler, all async operations
are handled within Cowboy processes.
