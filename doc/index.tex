\section{N2O: Web Framework for Erlang/OTP}

\paragraph{}
Nitrogen 2 Optimized, N2O was started as the first Erlang Web Framework
that relies completely on WebSocket transport. Great compatibility with Nitrogen
was retained and many improvements were made, such as binary page construction,
binary data transfer, all events transmitted over WebSocket channel, minumum process spawns,
work within Cowboy processes. N2O page renderer is several times faster
than the original one.

\subsection{Wide Coverage}
N2O is an unusual framework that solves problems in different web development domains
while staying small and concise at the same time. Started off as a Nitrogen concept
of server-side framework it can also build client-side offline applications
using same site sources. This becomes possible with the powerful Erlang JavaScript Parse
Transform which extends Erlang language to JavaScript platform with two-side
interoperability. You can as well use Elixir, LFE and Joxa languages to develop
your back-end.

\paragraph{}
It also combines DSL and HTML templates approaches allowing to build JavaScript
control elements using Erlang language and use inline rendering with DSL using
the same code for both client and server sides!
You decide how to use N2O, whether for mobile applications development using server-side rendering
for HTML and JavaScript saving CPU power on mobile devices or for creating rich desktop
offline applications with Erlang JavaScript compiler.

\newpage
\subsection*{Why Erlang in Web?}
We've measured all the existing modern web frameworks built with functional
languages and Cowboy is still the king. On the chart below you can see raw HTTP
performance of functional and C-like languages with concurrent
primitives (Go, D and Rust) on VAIO Z notebook with i7640M processor.

\includeimage{connections.png}{Web-Servers raw performance grand congregation}

\paragraph{}
Erlang was built for low latency binary data streaming in telecom systems.
It was designed with extreme manageability, scalability
and concurrent processing in mind. Assuming WebSocket channels are binary
telecom streams and web pages are user binary sessions
helps for better undestanding the choice of Erlang for the Web.

\paragraph{}
Using Erlang for the web allows you to unleash the full power of telecom systems for
web-scale, asynchronuous, non-blocking, sharded, event-driven,
message-passing, NoSQL, reliable, highly-available, high-performance,
secure, qualative, real-time, clusterable applications. See Erlang: The Movie II.

\paragraph{}
N2O outperforms full Nitrogen stack with only 2X raw HTTP Cowboy
performance downgrade thus upgrading rendering performance several
times compared to any other functional web framework. And
sure it's faster than raw HTTP node.js performance.

\subsection{Rich and Lightweight Applications}
There are two approaches for designing client/server communication.
The first one is called 'data-on-wire'. With this approach only JSON, XML or binary
data are transfered over RPC and REST channels. All HTML rendering
is performed on the client-side. This is the most suitable approach for desktop
applications. Examples include Chaplin/CoffeeScript, Meteor/JavaScript
and ClojureScript.

\paragraph{}
Another approach is sending pre-rendered parts of pages and JavaScript
and then only replacing HTML and executing JavaScript on the client-side. This approach
is better suited for mobile applications development since the client doesn't have much resources.

\paragraph{}
With N2O you can create both types of applications: using N2O REST framework
for the first type of application based on Cowboy REST API along with DTL
templates for initial HTML renderings. And also Nitrogen DSL-based approach
to model parts of the pages as widgets and control elements thanks to rich
Nitrogen elements collections provided by Nitrogen community. 

\paragraph{}
In cases where your system is built around Erlang infrastructure, N2O
is the best choice you can make for the fast web prototyping, simplicity
of use and codebase clarity. Despite HTML transfers over the wire,
you will still have access to all your Erlang services directly.

\paragraph{}
You can also create off-line clients using Erlang JavaScript compiler
just the way you'd use ClojureScript, JScala, Elm, WebSharper or similar
tools. N2O covers: REST micro-frameworks, server-side rendering,
client-side rendering, WebSocket events streaming, JavaScript generation
and JavaScript macro system along with {\bf AVZ} authorization
library (Facebook, Google, Twitter, Github, Microsoft), key-value storages
access library {\bf KVS} and {\bf MQS} Message Bus client library (GProc, RabbitMQ).

\subsection{JSON and BERT}
N2O uses JSON and BERT. All messages passed over
WebSockets are encoded with native Erlang External Term Format.
It is easy to parse it in JavaScript with {\bf Bert.decode(msg)}
and it helps to avoid complexity on server side. Please refer
to \footahref{http://bert-rpc.org}{http://bert-rpc.org} for more information.

\subsection{DSL and Templates}
We liked Nitrogen for the simple and elegant way of constructing typed
HTML with internal DSL which is analgous to Scala Lift,
OCaml Ocsigen and Haskell Blaze approach. It enables you developing re-usable control
elements and components in the host language.

\paragraph{}
Template-based approach requires programmers to deal with raw HTML,
like with Yesod, ASP, PHP, JSP, Rails, Yaws and ChicagoBoss. It allows
defining pages in terms of top-level controls, playholders
and panels. Thus N2O combines both approaches.

\paragraph{}
The main N2O advantage is fast prototyping. But we also use it for large
scale projects. Below is the complete Web Chat implementation using
WebSockets. It demonstrates Templates, DSL and async
interprocesses communication.

\newpage
\begin{lstlisting}[caption=chat.erl]
    -module(chat).
    -compile(export_all).
    -include_lib("n2o/include/wf.hrl").

    main() -> 
      [ #dtl { file="index", bindings=[{title,"Chat"},{body,body()}] } ].

    body() ->
        {ok,Pid} = wf:async(fun() -> loop() end),

      [ #span    { id=title,       body="Your chatroom name: " }, 
        #textbox { id=userName,    body="Anonymous" },
        #panel   { id=chatHistory, class=chat_history },
        #textbox { id=message },
        #button  { id=sendButton,  source=[userName,message],
                                   body="Send", 
                                   postback={chat,Pid} } ].

    event({chat,Pid}) ->
        Username = wf:q(userName),
        Message = wf:q(message),
        Terms = [ #span { text="Message sent" }, #br{} ],
        wf:reg(room),
        Pid ! {message, Username, Message}.

    loop() ->
        receive { message, Username, Message} ->
                  Terms = [ #span { body=Username }, ": ",
                            #span { body=Message }, #br{} ],
                  wf:insert_bottom(chatHistory, Terms),
                  wf:flush(room) end, loop().
\end{lstlisting}

You should try to build that with your favorite language/framework!

\subsection*{Changes from Nitrogen}
We feel free to break some of the compatability with the original
Nitrogen framework, mostly because we want to have a clean codebase
and fastest speed. However, it's still possible to easily port
Nitrogen web sites to N2O. E.g., N2O returns id and class semantics
of HTML and not {\bf html\_id}.

\paragraph{}
We simplified HTML rendering by not using
{\bf html\_encode} which should be handled by application layer.

\paragraph{}
Nitrogen.js originally created by Rusty Klophaus
was removed because of the pure WebSocket nature of N2O which doesn't
require XHR helper methods on the client. We added XHR fallback
handling by using 'Bullet' library written by Loïc Hoguin.

\paragraph{}
We also removed {\bf simple\_bridge} and optimized N2O at every level to
achieve the maximum performance and simplicity. We hope you'll enjoy
using N2O. We are fully convinced it's the most efficient way to
build Web applications in Erlang.

\paragraph{}
Original Nitrogen was tested in production under high-load and we
decided to remove {\bf nprocreg} process registry along 
with {\bf{action\_comet}} heavy process creation. N2O creates a single
process for async websocket handler, all async operations
are handled within Cowboy processes.
