\section{Actions}

\paragraph{}
{\bf \#action} record is base for all actions. It means that each action
has {\bf \#action} as its ancestor.

\vspace{1\baselineskip}
\begin{lstlisting}
    #action { ancestor,
              target,
              module,
              actions,
              source=[] }.
\end{lstlisting}
\vspace{1\baselineskip}

{\bf target} signify at which entity action will arise.

\subsection{JavaScript DSL {\bf \#jq}}
jQuery action provides mimics of JavaScript calls and assignments.
Depending on filling property or method field specific action performs.

\vspace{1\baselineskip}
\begin{lstlisting}
    -record(jq, {?ACTION_BASE(action_jq),
            property,
            method,
            args=[],
            right }).
\end{lstlisting}
\vspace{1\baselineskip}

Here is example of method calls:
\begin{lstlisting}
    wf:wire(#jq{target=n2ostatus,method=[show,select]}).
\end{lstlisting}
unfolded to calls:
\begin{lstlisting}
    $('#n2ostatus').show(); $('#n2ostatus').select();
\end{lstlisting}
\vspace{1\baselineskip}

And here is example of property chained assignments:
\begin{lstlisting}
    wf:wire(#jq{target=history,property=scrollTop,
        right=#jq{target=history,property=scrollHeight}}).
\end{lstlisting}
which transforms to:
\begin{lstlisting}
    $('#history').scrollTop = $('#history').scrollHeight;
\end{lstlisting}
\vspace{1\baselineskip}
Part of N2O API implemented using \#jq actions (updates and redirect).
This action is introduced as transitional in order to move
from Nitrogen DSL to using pure JavaScript transformations.

\newpage
\subsection*{Event Actions}
Everything that goes over WebSockets channel from server to client called actions.
Everything that goes on that channel from client to server called event. However
events itself are jQuery bindings on HTML elements and for performing those bindings
actions should be sent first. Such actions called event actions and there are three type
of them.

\subsection{Page Events \#event}
Page events are regular events that are routed to calling module. Postback used as main
routing parameter to {\bf event} module function. By providing {\bf source} elements list
values from HTML contorols gathered and packet for use with {\bf wf:q} accessor from page context.
Page event usually generated by active elements like {\bf \#button}, {\bf \#link},
{\bf \#textbox}, {\bf \#dropdown}, {\bf \#select}, {\bf \#radio} and others who contains postback field.

\subsection{Control Events \#control}
Control events dedicated to solve need of elements writers. When you developing your
own control elements usually you want event to be routed not to page but to element module.
For such purposes control event were introduced.

\subsection{API Events \#api}
When you need to call Erlang function from JavaScript directly you use API event.
API events routed to page module with {\bf api\_event/3} signature. API events were
used in {\bf AVZ} authorization library. Here is example how JSON login could be
implemented using api\_event:

\vspace{1\baselineskip}
\begin{lstlisting}
    api_event(appLogin, Args, Term) ->
        Struct = n2o_json:decode(Args),
        wf:info("Granted Access"),
        wf:redirect("/account").
\end{lstlisting}
\vspace{1\baselineskip}
And from JavaScript you call it like that:
\vspace{1\baselineskip}
\begin{lstlisting}
    document.appLogin(JSON.stringify(response));
\end{lstlisting}
\vspace{1\baselineskip}
All API events are binded to root of the HTML document.


\subsection{Message Box \#alert}
Message box {\bf alert} is very simple dialog useful for client debugging.
You can also use {\bf console.log} along with alerts.

\vspace{1\baselineskip}
\begin{lstlisting}
    event({debug,Var}) ->
        wf:wire(#alert{text="Debug: " ++ wf:to_list(Var)}),
\end{lstlisting}

\subsection{Confirmation Box \#confirm}
You can use confirmation boxes for simple approval with JavaScrip {\bf confirm} dialogs.
For custom dialogs you should extend this action. Confirmation box is just an example how to
organize such type of logic.

\vspace{1\baselineskip}
\begin{lstlisting}
    event(confirm) ->
        wf:wire(#confirm{text="Are you happy?",postback=continue}),

    event(continue) -> wf:info("Yes, you're right!");
\end{lstlisting}
\vspace{1\baselineskip}
