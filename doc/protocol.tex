\section{WebSocket Protocol}

\subsection*{Requests}

\subsection{Nitrogen Events {\bf pickle}}

Picled messaged are used if you send messages over unencrypted
channel and want to hide the content of the message,
that was generated on server. You can use BASE64 pickling mechanisms
with optional AES/RIPEMD160 encrypting.

\vspace{1\baselineskip}
\begin{lstlisting}
    ws.send(enc(tuple(atom('pickle'),
        binary('ddtake'),
        binary('g2gCaAVkAAJldmQABWluZGV4ZAAEdGFrZWsABH'+
                Rha2VkAAVldmVudGgDYgAABXViAAQKXmIAC3cK'),
        [tuple(atom('ddtake'),'0')])));
\end{lstlisting}
\vspace{1\baselineskip}

Where Base64 represents the N2O EVENT:

\vspace{1\baselineskip}
\begin{lstlisting}
    #ev{module=index,payload=take,trigger="take",name=event}
\end{lstlisting}
\vspace{1\baselineskip}

This is Nitrogen-based messaging model.
This request will return JSON with EVAL field only.

\subsection{Client Requests {\bf client}}

Client messages usually originated at client and represent the Client API Requests:

\vspace{1\baselineskip}
\begin{lstlisting}
    ws.send(enc(tuple(
        atom('client'),
        tuple(atom('join_game'),1000001))));
\end{lstlisting}
\vspace{1\baselineskip}

NOTE: This request may return JSON with EVAL and DATA fields.

\newpage
\subsection{Erlang Term Requests {\bf bert}}

When you want to receive \footahref{http://bert-rpc.org}{BERT} messages on client you usually ask the
server by sending:

\vspace{1\baselineskip}
\begin{lstlisting}
    ws.send(enc(tuple(
        atom('bert'),
        binary('API Request'));
\end{lstlisting}
\vspace{1\baselineskip}

This messages could be handled as this:

\vspace{1\baselineskip}
\begin{lstlisting}
    event({bert,Message}) ->
        wf:info("This API will return BERT enveloped echo"),
        {answer,echo,Message};
\end{lstlisting}
\vspace{1\baselineskip}

NOTE: you will always receive BERT messages.

\subsection{Binary Requests {\bf binary}}

When you need raw binary Blob on client side,
for images or other raw data you can ask server like this:

\vspace{1\baselineskip}
\begin{lstlisting}
    ws.send(enc(tuple(
        atom('binary'),
        binary('API Request'));
\end{lstlisting}
\vspace{1\baselineskip}

And handle also in binary clause:

\vspace{1\baselineskip}
\begin{lstlisting}
    event({binary,Message}) ->
        wf:info("This API will return Raw Binary"),
        <<84,0,0,0,108>>;
\end{lstlisting}
\vspace{1\baselineskip}

NOTE: if event returns not the binary client will recieve BERT encoded message.

\newpage
\subsection{Server Requests {\bf server}}

Server messages are usually being sent to client originated on the
server by sending <b>info</b> notifications directly to Web Socket process:

\vspace{1\baselineskip}
\begin{lstlisting}
    > WebSocketPid ! {server, Message}
\end{lstlisting}
\vspace{1\baselineskip}

You can obtain this Pid like here:

\vspace{1\baselineskip}
\begin{lstlisting}
    event(init) -> wf:info("Web Socket Pid: ~p",[self()]);
\end{lstlisting}
\vspace{1\baselineskip}

You can also send server messages from client relays:

\vspace{1\baselineskip}
\begin{lstlisting}
    ws.send(enc(tuple(
        atom('server'),
        binary('Binary Message'));
\end{lstlisting}
\vspace{1\baselineskip}

NOTE: This request may return JSON with EVAL and DATA fields.

\newpage
\subsection*{Responses}

\subsection{JSON enveloped EVAL and DATA}

Each message from Web Socket channel to Client encoded as JSON object.
\footahref{https://github.com/synrc/n2o\_scripts/blob/master/n2o/n2o.js}{N2O.js}
is used to decode WebSocket binary messages from JSON container.

\begin{lstlisting}
    { "eval": "ws.send("Send Back This String");",
      "data": [ 131,104,2,100,0,7,109,101,115,115,
                97,103,101,107,0,5,72,101,108,108,111 ] }
\end{lstlisting}

EVAL values evaluated immediately and DATA values passed
to handle\_web\_socket(data) function if exists.

\begin{lstlisting}
    function handle_web_socket(body) { console.log(body); }
\end{lstlisting}

\subsection{JSON enveloped BERT}

Usually in DATA come BERT messages (Binary Erlang Term Format).
\footahref{http://github.com/synrc/n2o\_scripts/blob/master/n2o/bert.js}{BERT.js}
is used to decode application protocol message.

\begin{lstlisting}
    function handle_web_socket(body) {
        console.log(String(dec(body))); }
\end{lstlisting}

\begin{lstlisting}
    E> Received: {message,"Hello"}
\end{lstlisting}

\subsection{BERT Binary}

Erlang RPC protocol interconnection with JavaScript nodes should be transfered as bert answrers.

\begin{lstlisting}
    function handle_web_socket(body) {
        console.log(String(dec(body))); }
\end{lstlisting}

\subsection{RAW Binary}

Raw images for fastest possible speed should be transfered as binary answers.

\begin{lstlisting}
    function handle_web_socket_blob(body) { }
\end{lstlisting}

\begin{lstlisting}
    E> Unknown Raw Binary Received: [72,101,108,108,111]
\end{lstlisting}
