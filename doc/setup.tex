\section{Setup}

\subsection{Prerequisites}
To run N2O sites you need Erlang R14 or higher and basho rebar installed.
N2O works on Windows, Mac and Linux.

\paragraph{}
You also need node.js, npm and  version greater that 0.8 to assemble client-side N2O applications.

\subsection{Kickstart Bootstrap}
To try N2O you just need to fetch it from Github and build. We don't use
fancy scripts so building process is OTP compatible: bootstrap site
is bundled as Erlang release.

\vspace{1\baselineskip}
\begin{lstlisting}
        $ git clone https://github.com/5HT/n2o
        $ cd n2o/samples
        $ rebar get-deps
        $ rebar compile
        $ ./nitrogen_static.sh
        $ ./release.sh
        $ ./release_sync.sh
        $ ./start.sh
\end{lstlisting}
\vspace{1\baselineskip}

Now you can try: \footahref{http://localhost:8000}{http://localhost:8000}.

\newpage
\subsection{Add dependency on N2O core}
If you want start using raw N2O core you can define N2O http and websocket handlers and cowboy static
handler in Cowboy dispatch parameter:

\begin{lstlisting}[caption=web\_sup.erl]
 cowboy:start_http(http, 100, [{port, wf:config(port)}],
                              [{env, [{dispatch, dispatch_rules()}]}]),

 dispatch_rules() ->
    cowboy_router:compile(
       [{'_', [
            {["/static/[...]"], cowboy_static,
                [{directory, {priv_dir, ?APP, [<<"static">>]}},
                 {mimetypes, {fun mimetypes:path_to_mimes/2, default}}]}, 
            {"/ws/[...]", bullet_handler, [{handler, n2o_bullet}]},
            {'_', n2o_cowboy, []}
    ]}]).
\end{lstlisting}

And add a minimal {\bf index.erl} page:

\vspace{1\baselineskip}
\begin{lstlisting}[caption=index.erl]
-module(index).
-compile(export_all).
-include_lib("n2o/include/wf.hrl").

main() -> #span{text=<<"Hello">>}.
\end{lstlisting}
\vspace{1\baselineskip}

\subsection*{Developer scripts for Sync}
During development we use some scripts which are needed for linking source
directories with release lib directories and also link BERT, N2O and jQuery javascript.
After making release you should run:

\vspace{1\baselineskip}
\begin{lstlisting}
        $ ./nitrogen_static.sh   # before release
        $ ./release_sync.sh      # after release
\end{lstlisting}
\vspace{1\baselineskip}

Now you can edit site sources and {\bf sync} will automaticaly recompile
and reload modules in release.
