
\section{Protocols}

We made this for you to be happy.
You can plug any binary formatting protocol like BERT, BED, MessagePack
or even textual ones like TXT, JSON or custom binary protocols like MQTT, AMQP or legacy XMPP.
Nitrogen-support here is just three protocol messages: {\bf pickle}, {\bf flush} and {\bf delivery}.
N2O provides also basic heartbeat protocol that can be formatted at your whim.
Currently N2O supports 4 protocols: Heartbeat, Nitrogen, Client/Server, Binary.

\vspace{1\baselineskip}
\begin{lstlisting}
    protocols() -> 
       wf:config(n2o,protocols,[ n2o_heart,
                                 n2o_nitrogen,
                                 n2o_client,
                                 n2o_binary  ]).

\end{lstlisting}
\vspace{1\baselineskip}

\includeimage{images/n2o_protocols.png}{Messaging Pipeline}  
\

For example in Skyline (DSL) application you use only nitrogen and binary:

\begin{lstlisting}
    > wf:config(n2o,protocols).
    [n2o_heart,n2o_nitrogen,n2o_binary]
\end{lstlisting}

And in Games (SPA) application you need only client/server protocol:

\begin{lstlisting}
    > wf:config(n2o,protocols).
    [n2o_heart,n2o_client]
\end{lstlisting}

\subsection{Heartbeat}

Heartbeat protocol is essential WebSocket application level protocol for
PING and N2O initialization. It ping every 4-5 seconds from client side to server
thus allowing to determine client online presence. On reconnection or initial connect
client sends N2O init marker telling to server to reinitialize the context.

\vspace{1\baselineskip}
\begin{lstlisting}
    ws.send('PING');
    ws.send('N2O,');
\end{lstlisting}
\vspace{1\baselineskip}

You can try manually send this messages in web console to see whats happening,
also you can enable logging the heartbeat protocol by including its module in log\_modules:

\vspace{1\baselineskip}
\begin{lstlisting}
    log_modules() -> [n2o_heart].
\end{lstlisting}
\vspace{1\baselineskip}

Heartbeat protocol PING request returns empty message NOP binary response;
N2O messages returns JSON with EVAL fields of rendered actions.

\subsection{Nitrogen Compatibility Layer}

Pickled messaged are used if you send messages over unencrypted
channel and want to hide the content of the message,
that was generated on server. You can use BASE64 pickling mechanisms
with optional AES/RIPEMD160 encrypting.

\vspace{1\baselineskip}
\begin{lstlisting}
    ws.send(enc(tuple(atom('pickle'),
        binary('ddtake'),
        binary('g2gCaAVkAAJldmQABWluZGV4ZAAEdGFrZWsABH'+
                Rha2VkAAVldmVudGgDYgAABXViAAQKXmIAC3cK'),
        [tuple(atom('ddtake'),'0')])));
\end{lstlisting}
\vspace{1\baselineskip}

Where Base64 represents the N2O EVENT:

\vspace{1\baselineskip}
\begin{lstlisting}
    #ev{module=index,payload=take,trigger="take",name=event}
\end{lstlisting}
\vspace{1\baselineskip}

This is Nitrogen-based messaging model. Nitrogen WebSocket processes receive also
flush and delivery protocol messages, but originated from server. These are internal Nitrogen
protocol messages. This request will return JSON with EVAL field only.

\subsection{Client/Server}

Client messages usually originated at client and represent the Client API Requests:

\vspace{1\baselineskip}
\begin{lstlisting}
    ws.send(enc(tuple(
        atom('client'),
        tuple(atom('join_game'),1000001))));
\end{lstlisting}
\vspace{1\baselineskip}

Server messages are usually being sent to client originated on the
server by sending {\bf info} notifications directly to Web Socket process:

\vspace{1\baselineskip}
\begin{lstlisting}
    > WebSocketPid ! {server, Message}
\end{lstlisting}
\vspace{1\baselineskip}

You can obtain this Pid like here:

\vspace{1\baselineskip}
\begin{lstlisting}
    event(init) -> wf:info("Web Socket Pid: ~p",[self()]);
\end{lstlisting}
\vspace{1\baselineskip}

You can also send server messages from client relays and vice versa.
But it is up to your application and client/server handlers how to handle these messages

\vspace{1\baselineskip}
\begin{lstlisting}
    ws.send(enc(tuple(
        atom('server'),
        tuple(atom('attach'),1000001))));
\end{lstlisting}
\vspace{1\baselineskip}

NOTE: client/server request may return JSON with EVAL and DATA fields.

\subsection*{JSON enveloped EVAL and DATA}

Each message from Web Socket channel to Client encoded as JSON object.
\footahref{https://github.com/5HT/n2o/blob/master/priv/n2o.js}{N2O.js}
is used to decode WebSocket binary messages from JSON container.

\begin{lstlisting}
    { "eval": "ws.send("Send Back This String");",
      "data": [ 131,104,2,100,0,7,109,101,115,115,
                97,103,101,107,0,5,72,101,108,108,111 ] }
\end{lstlisting}

EVAL values evaluated immediately and DATA values passed
to handle\_web\_socket(data) function if exists.

\begin{lstlisting}
    function handle_web_socket(body) { console.log(body); }
\end{lstlisting}

\subsection*{JSON enveloped BERT}

Usually in DATA come BERT messages (Binary Erlang Term Format).
\footahref{https://github.com/5HT/n2o/blob/master/priv/bert.js}{BERT.js}
is used to decode application protocol message.

\begin{lstlisting}
    function handle_web_socket(body) {
        console.log(String(dec(body))); }
\end{lstlisting}

\begin{lstlisting}
    E> Received: {message,"Hello"}
\end{lstlisting}

\subsection{Binary}

When you need raw binary Blob on client side,
for images or other raw data you can ask server like this:

\vspace{1\baselineskip}
\begin{lstlisting}
    ws.send(enc(tuple(
        atom('binary'),
        binary('API Request'));
\end{lstlisting}
\vspace{1\baselineskip}

And handle also in binary clause:

\vspace{1\baselineskip}
\begin{lstlisting}
    event({binary,Message}) ->
        wf:info("This API will return Raw Binary"),
        <<84,0,0,0,108>>;
\end{lstlisting}
\vspace{1\baselineskip}

NOTE: if event returns not the binary client will receive BERT encoded message.

\subsection*{BERT}

Erlang RPC protocol interconnection with JavaScript nodes should be transferred as bert answers.

\begin{lstlisting}
    function handle_web_socket(body) {
        console.log(String(dec(body))); }
\end{lstlisting}

\subsection*{RAW Binary}

Raw images for fastest possible speed should be transferred as binary answers.

\begin{lstlisting}
    function handle_web_socket_blob(body) { }
\end{lstlisting}

\begin{lstlisting}
    E> Unknown Raw Binary Received: [72,101,108,108,111]
\end{lstlisting}
 
